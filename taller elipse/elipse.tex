\documentclass{article}
\begin{document}
\title{Taller: Elementos de la Elipse}
\author{Mateo Peñaranda Newman \\ Juan Pablo Sojo}
\date{Agosto 2025}
\maketitle

\section{Reconocimiento de Elementos}
Define con tus palabras qué es el:
\subsection{Centro}
Es el punto donde cortan el eje mayor y eje menor. La distancia del centro a cualquiera de los focos se representa como c.
\subsection{Focos}
Son dos puntos equidistantes al centro. La suma de las distancias entre cualquier punto del elipse y los focos es constante.
\subsection{Puntos B}
Son los puntos donde el eje menor corta con el elipse.
\subsection{Eje mayor}
Es el mayor de los dos segmentos perpendiculares que definen al elipse, sobre el cual se encuentran los focos. Su longitud se representa como 2a.
\subsection{Eje menor}
Es el menor de los dos segmentos perpendiculares que definen al elipse. Su longitud se representa como 2b y se relaciona a las otras distancias mediante la ecuación $a^2=b^2+c^2$

\section{Identificación en ecuaciones}
\subsection{Para la elipse}
$$(\frac{x^2}{25})+(\frac{y^2}{9})=1$$
\subsubsection{Identifica el foco, centro, puntos B y los ejes}
\begin{equation}
    \begin{aligned}
        
    \end{aligned}
\end{equation}

\end{document}