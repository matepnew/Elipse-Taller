\documentclass{article}
\begin{document}
\title{Taller: Elementos de la Elipse}
\author{Mateo Peñaranda Newman \\ Juan Pablo Sojo}
\date{Agosto 2025}
\maketitle

\section{Reconocimiento de Elementos}
Define con tus palabras qué es el:
\subsection{Centro}
Es el punto donde cortan el eje mayor y eje menor. La distancia del centro a cualquiera de los focos se representa como c.
\subsection{Focos}
Son dos puntos equidistantes al centro. La suma de las distancias entre cualquier punto del elipse y los focos es constante.
\subsection{Puntos B}
Son los puntos donde el eje menor corta con el elipse.
\subsection{Eje mayor}
Es el mayor de los dos segmentos perpendiculares que definen al elipse, sobre el cual se encuentran los focos. Su longitud se representa como 2a.
\subsection{Eje menor}
Es el menor de los dos segmentos perpendiculares que definen al elipse. Su longitud se representa como 2b y se relaciona a las otras distancias mediante la ecuación $a^2=b^2+c^2$
\subsection{Distancia focal}
Es la distancia entre los dos focos, representada como $2c$
\subsection{Lado recto}
Son los segmentos perpendiculares al eje mayor que pasan por uno de los focos y tienen sus extremos sobre la elipse. Su longitud es dada por $\frac{2b^2}{a}$
\section{Identificación en ecuaciones}
\subsection{Para la elipse}
$$(\frac{x^2}{25})+(\frac{y^2}{9})=1$$
\subsubsection{Identifica el foco, centro, puntos B y los ejes}
De la ecuación se tiene que 
$$a=\sqrt{25}=5 , b=\sqrt{9}=3$$
$$c=\sqrt{a^2-b^2}=\sqrt{16}=4$$

Centro $=(0,0)$

Focos $$F=(4,0)$$ $$F'=(-4,0)$$

Puntos B $$B=(0,3)$$ $$B'=(0,-3)$$

Eje mayor $=2a=10$

Eje menor $=2b=6$
\subsubsection{Especifica la longitud del eje mayor y menor}

Eje mayor $=2a=10$

Eje menor $=2b=6$
\subsubsection{Determina la distancia focal}
Distancia focal $=2c=8$

\subsection{Para la elipse}
$$(\frac{x^2}{9})+(\frac{y^2}{16})=1$$
\subsubsection{Identifica si es horizontal o vertical}
Su eje mayor es paralelo al eje $y$ (es veritcal), ya que $a$ está en el término con $y$
\subsubsection{Escribe las coordenadas de todos sus elementos (centro, vértices, focos, puntos B)}
De la ecuación se tiene que 
$$a=\sqrt{16}=4 , b=\sqrt{9}=3$$
$$c=\sqrt{a^2-b^2}=\sqrt{7}$$

Centro $=(0,0)$

Focos $$F=(0,\sqrt{7})$$ $$F'=(0,-\sqrt{7})$$

Puntos B $$B=(3,0)$$ $$B'=(-3,0)$$

Vértices $$A=(0,4)$$ $$A'=(0,-4)$$ $$B=(3,0)$$ $$B'=(-3,0)$$

\section{Aplicación y razonamiento}
\subsection{Explica por qué el eje mayor también se llama eje focal.}
Se llama así porque es tanto la recta sobre la que se encuentra el "diámetro" mayor, como sobre la que se encuentran los dos focos.
\subsection{Explica con tus palabras qué representa el lado recto y cómo se obtiene.}
Es el segmento de recta perpendicular al eje focal que pasa por uno de los focos y limita con el borde de la elipse. Se obtiene con la fórmula $\frac{2b^2}{a}$
\end{document}